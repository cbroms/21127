\documentclass[12pt]{article}
\usepackage{amssymb}
\usepackage{multicol}
\usepackage{graphicx}
\usepackage{color}
\usepackage{amsthm}
\usepackage{hyperref}
\usepackage{amsmath}
\usepackage{verbatim}
\usepackage{caption}
\usepackage{datetime}
\usepackage{subcaption}
\usepackage{amsmath, epsfig}
\usepackage[latin1]{inputenc}
\usepackage{enumitem}
\newtheorem{theorem}{Theorem}
\newtheorem{lemma}{Lemma}
\newtheorem{corollary}{Corollary}
\newcommand{\abs}[1]{\left| #1\right|}
\newcommand{\Lap}[1]{\mathcal{L}\left\{#1\right\}}
\newcommand{\solution}[1]{
\color{red}\begin{quote}Solution:\quad 
\color{black} #1
\end{quote}\color{black}
}
\newcommand{\ba}{\backslash}
\newcommand{\Ber}{\hbox{Ber}}
\DeclareMathOperator{\Gcd}{gcd}
\renewcommand{\gcd}[2]{\Gcd\left(#1, #2\right)}
\newcommand{\p}[1]{\mathbb{P}\left(#1\right)}
\newcommand{\e}[1]{\mathbb{E}(#1)}
\newcommand{\Po}[1]{\hbox{Po}(#1)}
\newcommand{\var}[1]{\hbox{Var}(#1)}
\newcommand{\Z}{\mathbb{Z}}
\newcommand{\R}{\mathbb{R}}
\newcommand{\Q}{\mathbb{Q}}
\newcommand{\N}{\mathbb{N}}
\newcommand{\floor}[1]{\left\lfloor #1\right\rfloor}
\newcommand{\C}{\mathbb{C}}
\DeclareMathOperator{\Diam}{diam}
\newcommand{\diam}[1]{\Diam\left(#1\right)}
\renewcommand\qedsymbol{$\blacksquare$}

\newcommand{\Mod}[1]{\ (\bmod\ #1)}

\begin{document}
\title{21-127 Homework 11
}
\author{Christian Broms \\ Section J}
\date{\today}
\maketitle
Complete the following problems. Fully justify each response.


\begin{enumerate}

\item Let $(X, 0, 1, +, \cdot)$ be a field, where $X$ is a finite set. Prove that there is no ordering $\leq$ on $X$ under which $X$ is an ordered field.

\begin{proof}
If $X$ is finite, then $|X| = b$. We know that $0 \leq 1$, and by axiom F1 that $0\neq1 \Rightarrow 0<1$. By extension, it holds that $a < a+1$ by OF2 for some $a \in X$. However, since $X$ is finite, that means that there exists one element that cannot be compared with anything; the maximum of $X$. Consider the field ordering 
$$0 < 1 < 1+1 < 1+1+1< 1+1+1+\ldots$$
Since $X$ is finite, the final element in the ordering is the maximum. Let this max value be $b$. Thus, $b < b+1$, but notice that $b+1 \notin X$ since the set is finite. So to be ordered, it must hold that $b \leq a$ for some $a \in X$. But we have shown that $b\neq a$ by F1. And $b \not< a$ because by definition $b$ is the maximum value in $X$. Thus, $X$ cannot be an ordered field since $b + 1\notin X$ and $b$ cannot be compared to any element. 
\end{proof}

\item Let $(X, 0, 1, +, \cdot, \leq)$ be an ordered field. Prove each of the following basic ordered field properties, from axioms.
\begin{enumerate}
\item For all $x\in X$, $x^2>0$.
\begin{proof}
Assuming $x \neq 0$, 
there are two cases, $x>0$ or $x<0$. 

Case 1: $x>0$. We use the fact that $x \cdot 0 = 0$, as previously proven. So,
$$x >0$$
$$x \cdot x > x \cdot 0$$
$$x^2 > 0$$

Case 2: $x < 0$. We use the additive inverse and the fact above. So, 
$$-x > 0$$ 
$$-x \cdot -x > 0 \cdot -x$$ 
$$x^2 > 0$$ 
and we are done. 
\end{proof}
\item For all $w, x, y, z\in X$, if $w\leq x$ and $y\leq z$, then $w+y\leq x+z$.
\begin{proof}
If $w\leq x$ and $y\leq z$, then we have $z-y \geq 0$ and $x-w \geq 0$. Rearranging, we get $w-x \leq 0$ and $z - y \geq 0$. Thus, it follows that 
$$z-y \geq w-x$$
$$z \geq w - x + y$$
$$z + x \geq w+y$$
Thus, $w+y\leq x+z$.
and we are done. 
\end{proof}
\item For all $x, y, z\in X$, if $x\geq 0$ and $y\leq z$, then $xy\leq xz$.
\begin{proof}
Since $x \geq 0$ and $y \leq z$ then $y-z \leq 0$. So, 
$$x \cdot (y-z) \leq 0 $$
$$xy - xz \leq 0 $$
$$xy \leq xz $$
and we are done. 
\end{proof}
\item For all $x, y, z\in X$, if $x\leq 0$ and $y\leq z$, then $xy\geq xz$.
\begin{proof}
Since $x\leq 0$ and $y\leq z$, then $y-z \leq 0$. Then since $x\leq 0$, we have two negative numbers and thus
$$x \cdot (y-z) \geq 0 \cdot x$$
$$xy- xz \geq 0$$
$$xy \geq xz$$
 and we are done. 
\end{proof}
\end{enumerate}

\item Let $\vec{x}, \vec{y}, \vec{z}\in \R^n$. Prove that $\|\vec{x}-\vec{z}\|\leq \|\vec{x}-\vec{y}\|+\|\vec{y}-\vec{z}\|$.

\begin{proof}
Recall the triangle inequality: $\|a + b\| = \|a\| + \|b\|$. Let $a = \vec{x} - \vec{y}$ and let $b= \vec{y} - \vec{z}$. Then we have $$\|\vec{x}-\vec{y} + \vec{y} - \vec{z}\|\leq \|\vec{x}-\vec{y}\|+\|\vec{y}-\vec{z}\|$$ $$\|\vec{x}-\vec{z}\|\leq \|\vec{x}-\vec{y}\|+\|\vec{y}-\vec{z}\|$$ and we are done.  

\end{proof}

\item Let $x_n = \frac{n+2}{n+1}$. Prove that $x_n$ converges to $1$.

\begin{proof}
Let $\epsilon > 0$ and let $N \in \N$ with $N > \frac{1}{\epsilon} - 1$. Let $n \geq N$. Then $|\frac{n+2}{n+1} - 1| = \frac{1}{n+1} < \frac{1}{N + 1} < \frac{1}{\frac{1}{\epsilon} - 1 + 1} = \epsilon$. Therefore, by definition, $(\frac{n+2}{n+1}) \to 1$.
\end{proof}

\item Let $(x_n)$ and $(y_n)$ be sequences of real numbers, with $(x_n)\to a$ and $(y_n)\to b$. Let $z_n=x_ny_n$ for all $n\in \N$. Prove that $(z_n)\to ab$.

\begin{proof}
Let $\epsilon > 0$. Since every convergent sequence of real numbers must be bounded, there exists some $M > 0$, $N_1 \in \N$, where 
$$\forall n \geq N_1, |x_n| < M$$
In addition, since $(x_n)$ and $(y_n)$ both converge, then there exists $N_2, N_3 \in \N$ such that 
$$\forall n \geq N_2, |x_n - a| < \frac{\epsilon}{2|b|}$$
$$\forall n \geq N_3, |y_n - b| < \frac{\epsilon}{2M}$$

So $\forall n \geq N$, $N = \max\{N_1, N_2, N_3\}$.

Now, $$|(x_n - y_n) - ab| = |x_ny_n - x_nb + x_nb - ab|$$
$$\leq |x_ny_n -x_nb| + |x_nb-ab|$$
$$\leq |x_n(y_n-b)| + |b(x_n-a)|$$
Substituting, we have
$$M \cdot \frac{\epsilon}{|2M|} + |b| \cdot \frac{\epsilon}{|2b|} = \frac{\epsilon}{2} +\frac{\epsilon}{2} = \epsilon$$

Hence, $\forall \epsilon > 0$, $\exists N \in \N$ such that if $n \geq N$, $|x_ny_n - ab| < \epsilon$.
Thus, we conclude $x_ny_n\to ab$.
\end{proof}

\item Prove that if $(x_n)$ is a monotonically decreasing sequence, having a lower bound, then $(x_n)$ converges.

\begin{proof}
First, we must show that $x_n$ has a greatest lower bound; that it has an infimum. 

Consider the set bounded between two values, given by $X = \{x\in \R\ |\ m < x < n\}$, where $m, n \in \R$. So, we have that $m \leq x\ \forall x \in X$. Hence, $m$ is by defintion a lower bound. Now we show that for some $b$, if $m < b$ then $\exists a \in X$ so $a < b$, to prove $m$ is the greatest lower bound. There are two cases, either $m < b < n$ or $n \leq b$. In both cases, we need to choose values $a \in X$ such that $a <b$. 

In the first case, we set $a = \frac{b+m}{2}$. In the second case, set $a = \frac{n+m}{2}$. 
Notice, in both cases, $a <b$ holds, so the properties of infimum are fufilled. Therefore, we conclude that the infimum of $X$ is $m$. 

Now, we know that any arbitrary bounded set has an greatest lower bound. 

Next, we show that $x_n$ converges.

Since $\R$ is complete, $\{x_n\ |\ n \in \N\}$ has a greatest lower bound, say $a$. Fix $\epsilon > 0$. Note that by defintion, $a + \epsilon$ is not a lower bound for $x_n$ since $a$ is the greatest lower bound. Thus, there must be some $x_n$ having $x_N < a + \epsilon$. For any $n \geq N$, we therefore have
$$a - \epsilon < a \leq x_n \leq x_N < a + \epsilon$$
Thus, $a-\epsilon < x_n < a + \epsilon$, so $|x_n - a| < \epsilon$, so $(x_n) \to a$. Therefore, we conclude that the monotonically decreasing bounded sequence $x_n$ converges. 
\end{proof}

\end{enumerate}
\end{document}