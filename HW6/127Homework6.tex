\documentclass[12pt]{article}
\usepackage{amssymb}
\usepackage{multicol}
\usepackage{graphicx}
\usepackage{color}
\usepackage{amsthm}
\usepackage{hyperref}
\usepackage{amsmath}
\usepackage{verbatim}
\usepackage{caption}
\usepackage{datetime}
\usepackage{subcaption}
\usepackage{amsmath, epsfig}
\usepackage[latin1]{inputenc}
\usepackage{enumitem}
\newtheorem{theorem}{Theorem}
\newtheorem{lemma}{Lemma}
\newtheorem{corollary}{Corollary}
\newcommand{\abs}[1]{\left| #1\right|}
\newcommand{\Lap}[1]{\mathcal{L}\left\{#1\right\}}
\newcommand{\solution}[1]{
\color{red}\begin{quote}Solution:\quad 
\color{black} #1
\end{quote}\color{black}
}
\newcommand{\ba}{\backslash}
\newcommand{\Ber}{\hbox{Ber}}
\DeclareMathOperator{\Gcd}{gcd}
\renewcommand{\gcd}[2]{\Gcd\left(#1, #2\right)}
\newcommand{\p}[1]{\mathbb{P}\left(#1\right)}
\newcommand{\e}[1]{\mathbb{E}(#1)}
\newcommand{\Po}[1]{\hbox{Po}(#1)}
\newcommand{\var}[1]{\hbox{Var}(#1)}
\newcommand{\Z}{\mathbb{Z}}
\newcommand{\R}{\mathbb{R}}
\newcommand{\Q}{\mathbb{Q}}
\newcommand{\N}{\mathbb{N}}
\newcommand{\floor}[1]{\left\lfloor #1\right\rfloor}
\newcommand{\C}{\mathbb{C}}
\DeclareMathOperator{\Diam}{diam}
\newcommand{\diam}[1]{\Diam\left(#1\right)}
\renewcommand\qedsymbol{$\blacksquare$}

\begin{document}
\title{21-127 Homework 6}
\author{Christian Broms \\ Section J}
\date{\today}
\maketitle

Complete the following problems. Fully justify each response.

\begin{enumerate}

\item Let $a, b\in \Z$. Prove that if $d, d'$ are both gcds of $a$ and $b$, then $d=\pm d'$.

\begin{proof}
Since $d, d'$ are both gcds of $a$ and $b$, then by definition, $a|d'$, $b|d'$, $a|d$, $b|d$, so $d|d'$ and $d'|d$ because both are GCDs. When $d$, $d' = 0$, then the case is trivial and we can say $0 = \pm0$, or if $d$ or $d'$ is zero then the other must also be 0 since both are gcds of $a, b$ and again we say $0 = \pm0$. In other cases, since $d|d'$ and $d'|d$ we know $\abs{d} \leq \abs{d'}$ and $\abs{d'} \leq \abs{d}$ when $d$, $d' \neq 0$. Therefore, $\abs{d} = \abs{d'}$. Because this relation is reliant on absolute value, inserting $-d'$ or $d'$ will yield the same result. Thus, we conclude $d=\pm d'$.
\end{proof}

\item Let $a, b\in \Z$, and let $d=\gcd{a}{b}$. Prove that $\frac{a}{d}$ and $\frac{b}{d}$ are coprime.

\begin{proof}
Because $d=\gcd{a}{b}$, we know there exist $u, v \in \Z$ such that $d = au + bv$, and $\frac{au}{d} + \frac{bv}{d}=1$. So, by Bezout's Lemma, we can say $\gcd{\frac{a}{d}}{\frac{b}{d}} = 1$. Thus, we have shown that $\gcd{\frac{a}{d}}{\frac{b}{d}}$ is coprime. 
\end{proof}

\item Let $a, b\in \Z$. Prove that there exists a unique positive least common multiple of $a$ and $b$. (Note: here you must prove both existence and uniqueness.)

\begin{proof}
Existence. Let $X$ be the set of common multiples of $a$ and $b$, defined as $X = \{x \in \N\ |\ a|x, b|x\}$. We know $X \neq \emptyset$ because $\abs{ab} \in X$. Thus, by the well ordering principle, there is a smallest element in $X$. Let this smallest element be $m$, so $a|m$ and $b|m$. So let $n$ be such that $a | n$ and $b | n$. We need to show that $m | n$ to fufill the second property of the LCM. 

 Then $n = qm + r$ so $n - qm = r$ with $0 \leq r < m$. Because $a |m$ and $a |m$ then $a | r$. The same is true for $b$, so $b | r$. Thus, $r = 0$, because if $r > 0$, then $r \in X$, though this is impossible because $ r < m$ and we established that $m$ is the smallest element in the set. Therefore, $r$ must be 0, and we conclude that $n = qm$ or $m|n$ and hence the LCM exists. 

Uniqueness. Assume that $k$ and $\ell$ are both Least Common Multiples of $a$ and $b$. Then, $a|\ell$ and $b|\ell$ and $a|k$ and $b|k$. In addition, we know that $k|\ell$ and $\ell|k$ because both are LCMs. Then, $\abs{k} \leq \abs{\ell}$ and $\abs{\ell} \leq \abs{k}$. Therefore, $\abs{k} = \abs{\ell}$, and it follows that there can only be one unique LCM. 
\end{proof}


\item Let $p\in \Z$. Prove that the following are equivalent
\begin{enumerate}
\item $p$ is irreducible.
\item The only divisors of $p$ are $\pm 1, \pm p$
\item $p$ is prime (under the definition in Section 3.2, that $p$ is prime whenever $p|ab\Rightarrow p|a\vee p|b$).


\begin{proof}
(a$\Rightarrow$b). Let $a,b \in \Z$, and assume $p$ is irreducible and $a | p$. Then $p = ab$. Because $p$ is irreducible, then either $a$ or $b = \pm 1$, that is either $a$ or $b$ is a unit. We can write $a,b$ in terms of $p$, so if $a$ is a unit then $b = \pm p$, and if $b$ is a unit then $a = \pm p$. Thus, we conclude the only divisors of $p$ are $\pm 1, \pm p$.\hfill \break

(b$\Rightarrow$c). Assume the only divisors of $p$ are $\pm 1, \pm p$. Then we can say $p = ab$. Without loss of generality, say $a = \pm 1$ and $b = \pm p$, though the order does not matter. Then, $p | ab \Rightarrow p | 1$ or $p | p$. This fits the definition of a prime, so we conclude that $p$ is a prime. 
\hfill \break

(c$\Rightarrow$a). Assume $p$ is prime. Then we can say $p |ab$ and $p|a$ or $p|b$ for some $a,b \in \Z$. First, consider the case when $p|a$. We can say $a = kp$ for $k \in \Z$. Then, $p = ab = kpb$ and $0 = p(1-kb)$, which implies $0 = 1 - kb$, and $b = 1$, so $b$ is a unit. Therefore, since $b$ is a unit, we know that $p$ is irreducible. The same is true when considering $p|b$. 
\end{proof}
\end{enumerate}

\item Suppose $p_1, p_2, \dots, p_r\in \Z$ are primes. Let $a=p_1^{k_1}p_2^{k_2}\dots p_r^{k_r}$, and let $b=p_1^{\ell_1}p_2^{\ell_2}\dots p_r^{\ell_r}$, where $k_1, k_2, \dots k_r, \ell_1, \ell_2, \dots, \ell_r$ are nonnegative integers. Prove that $\gcd{a}{b} = p_1^{m_1}p_2^{m_2}\dots p_r^{m_r}$, where $m_i = \min\{k_i, \ell_i\}$ for all $1\leq i\leq r$.

\begin{proof}

We can write $a,b$ as the product of their GCD and some other integer $\alpha, \beta \in \Z$, such that $a = (p_1^{m_1}p_2^{m_2}\dots p_r^{m_r})\alpha$ and $b = (p_1^{m_1}p_2^{m_2}\dots p_r^{m_r})\beta$, where $p_1^{m_1}p_2^{m_2}\dots p_r^{m_r}$ is the GCD of $a$ and $b$. Note, $\gcd{\alpha}{\beta} = 1$ because the largest prime factors of $a, b$ are contained within their GCD. Thus, we write $\alpha = p_1^{k_1 - m_1}p_2^{k_2 - m_2}\dots p_r^{k_r - m_r}$ and $\beta = p_1^{\ell_1 - m_1}p_2^{\ell_2 - m_2}\dots p_r^{\ell_r - m_r}$. Because $\gcd{\alpha}{\beta} = 1$ we know that either $k_i - m_i = 0$ or $\ell_i - m_i = 0$. Rearranging, we have $k_i = m_i$ or $\ell_i = m_i$. Therefore, $m_i$ must be the lesser of $k_i$ and $\ell_i$ because $k_i - m_i$, $\ell_i - m_i \geq 0$. Hence, $m_i = \min\{k_i, \ell_i\}$, and we have shown $\gcd{a}{b} = p_1^{m_1}p_2^{m_2}\dots p_r^{m_r}$, where $m_i = \min\{k_i, \ell_i\}$ for all $1\leq i\leq r$. 

\end{proof}

\item Prove that for all $n\geq 2$, there exists a prime in the set \[\{k\in \Z\ | \ n\leq k \leq n!\}.\] (Hint: consider the divisors of $n!-1$. Can they be in the set $\{1, 2, \dots, n\}$?)

\begin{proof}
We have $n \leq k \leq (n! - 1)$. There are two possiblities. In the first, $(n! - 1)$ is prime, and we can say $k = (n! - 1)$, where $n \leq k \leq (n! - 1)$. We then know that there exists a prime in the set $\{k\in \Z\ | \ n\leq k \leq n!\}.$ In the second case, $(n! - 1)$ is not prime, and it has no factors between 2 and $n$. This is because by definition $n!$ has divisors 2 through $n$. Since $n!$ and $(n! - 1)$ are coprime, $(n!-1)$ cannot have a common divisor. Because we can prime factorize $(n! - 1)$, and it has no factors between 2 and $n$, then the prime factors must be between $n$ and $(n! - 1)$. Thus we know that there exists a prime in the set $\{k\in \Z\ | \ n\leq k \leq n!\}.$
\end{proof}

\end{enumerate}

\end{document}