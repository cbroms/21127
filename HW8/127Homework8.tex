\documentclass[12pt]{article}
\usepackage{amssymb}
\usepackage{multicol}
\usepackage{graphicx}
\usepackage{color}
\usepackage{amsthm}
\usepackage{hyperref}
\usepackage{amsmath}
\usepackage{verbatim}
\usepackage{caption}
\usepackage{datetime}
\usepackage{subcaption}
\usepackage{amsmath, epsfig}
\usepackage[latin1]{inputenc}
\usepackage{enumitem}
\newtheorem{theorem}{Theorem}
\newtheorem{lemma}{Lemma}
\newtheorem{corollary}{Corollary}
\newcommand{\abs}[1]{\left| #1\right|}
\newcommand{\Lap}[1]{\mathcal{L}\left\{#1\right\}}
\newcommand{\solution}[1]{
\color{red}\begin{quote}Solution:\quad 
\color{black} #1
\end{quote}\color{black}
}
\newcommand{\ba}{\backslash}
\newcommand{\Ber}{\hbox{Ber}}
\DeclareMathOperator{\Gcd}{gcd}
\renewcommand{\gcd}[2]{\Gcd\left(#1, #2\right)}
\newcommand{\p}[1]{\mathbb{P}\left(#1\right)}
\newcommand{\e}[1]{\mathbb{E}(#1)}
\newcommand{\Po}[1]{\hbox{Po}(#1)}
\newcommand{\var}[1]{\hbox{Var}(#1)}
\newcommand{\Z}{\mathbb{Z}}
\newcommand{\R}{\mathbb{R}}
\newcommand{\Q}{\mathbb{Q}}
\newcommand{\N}{\mathbb{N}}
\newcommand{\floor}[1]{\left\lfloor #1\right\rfloor}
\newcommand{\C}{\mathbb{C}}
\DeclareMathOperator{\Diam}{diam}
\newcommand{\diam}[1]{\Diam\left(#1\right)}
\renewcommand\qedsymbol{$\blacksquare$}

\newcommand{\Mod}[1]{\ (\bmod\ #1)}


\begin{document}
\title{21-127 Homework 8
}
\author{Christian Broms \\ Section J}
\date{\today}
\maketitle

Complete the following problems. Fully justify each response.

NOTE: due to the Spring Break, this homework set is a bit longer than is typical. You only need to turn in those problems marked with (*).

\begin{enumerate}


\item (*) Let $X$ be a finite set, and suppose there is a surjection $f:X\to Y$. Prove that $|X|\geq |Y|$. 
\begin{proof}
If $f$ is surjective, then $\forall y \in Y$, $\exists x \in X$ such that $f(x) = y$. Since $f$ is a function, each $x$ contributes at most one to $y$. Let $|B| = n$. There must be at least $n$ $x$s, so $|X| \geq n = |Y|$.
\end{proof}

\item (*) Let \[X_2=\{n\ | 1\leq n\leq 200, n=k^2\ \exists k\in \Z\},\] \[X_3=\{n\ | \ 1\leq n\leq 200, n=k^3\ \exists k\in \Z\},\] and \[X_4 = \{n\ | \ 1\leq n\leq 200, n=k^4\ \exists k\in \Z\}.\] Determine $|X_2\cup X_3\cup X_4|$.

\item (*) Let $X=\{ (a_1, a_2, \dots, a_n)\ | \ a_i\in\{0,1\}\forall i\} = \{0,1\}^n$. These are sometimes called bitstrings of length $n$.
\begin{enumerate}
\item Show that there is a bijection between $X$ and $\{f:[n]\to\{0,1\}\}$, the set of functions from $[n]$ to $\{0,1\}$
\item Show that there is a bijection between $X$ and $\mathcal{P}([n])$.
\item Determine $|X|$.
\end{enumerate}

\item (*) Let $X$ and $Y$ be finite sets. Define $X^Y = \{f:Y\to X\}$, the set of functions from $Y$ to $X$. Prove that $|X^Y|=|X|^{|Y|}$.

\item (*) Let $n, k\in \N$ with $n\geq k$. Prove, by counting in 2 ways, that $k{n\choose k} = (n-k+1){n\choose k-1}$.

\begin{proof}
Suppose we wish to select from a group of $n$ people a committee of $k$ people with a president (so there are $k-1$ members plus one president on the committee). There are two ways to do this. First, we could select the $k$ people for the committee from the $n$ total people, and then select the president from the committee of $k$ people so we have ${k \choose 1}{n\choose k} = k{n\choose k}$.

On the other hand, we could first select the committee with $k - 1$ members from the $n$ total people. We then choose the president from the remaining group of people, which is now $n - k + 1$ in size. Using this selection technique we get ${n-k+1 \choose 1}{n\choose k-1} = (n-k+1){n\choose k-1}$. 

Thus, since both sides of the equality count the same set, they are equal. 
\end{proof}

\item (*) How many subsets of $[20]$ contain a multiple of 4? Prove that your answer is correct.

\begin{proof}
There are $2^{20}$ possible subsets of $\{1, 2, 3, \ldots 20\}$. There are $2^5$ possible subsets of $\{4, 8, 12, 16, 20\}$. Thus, there are $2^5 = 32$ subsets of $[20]$ that contain a multiple of 4.  
\end{proof}

\item (*) Let $f:X\to Y$ be a bijection. Prove that $X$ is countably infinite if and only if $Y$ is countably infinite.

\item (*) Let $X$ be a finite set. Show that $\N^X=\{f:X\to \N\}$ is countably infinite.

\end{enumerate}

\end{document}