\documentclass[12pt]{article}
\usepackage{amssymb}
\usepackage{multicol}
\usepackage{graphicx}
\usepackage{color}
\usepackage{amsthm}
\usepackage{hyperref}
\usepackage{amsmath}
\usepackage{verbatim}
\usepackage{caption}
\usepackage{datetime}
\usepackage{subcaption}
\usepackage{amsmath, epsfig}
\usepackage[latin1]{inputenc}
\usepackage{enumitem}
\newtheorem{theorem}{Theorem}
\newtheorem{lemma}{Lemma}
\newtheorem{corollary}{Corollary}
\newcommand{\abs}[1]{\left| #1\right|}
\newcommand{\Lap}[1]{\mathcal{L}\left\{#1\right\}}
\newcommand{\solution}[1]{
\color{red}\begin{quote}Solution:\quad 
\color{black} #1
\end{quote}\color{black}
}
\newcommand{\ba}{\backslash}
\newcommand{\Ber}{\hbox{Ber}}
\DeclareMathOperator{\Gcd}{gcd}
\renewcommand{\gcd}[2]{\Gcd\left(#1, #2\right)}
\newcommand{\p}[1]{\mathbb{P}\left(#1\right)}
\newcommand{\e}[1]{\mathbb{E}(#1)}
\newcommand{\Po}[1]{\hbox{Po}(#1)}
\newcommand{\var}[1]{\hbox{Var}(#1)}
\newcommand{\Z}{\mathbb{Z}}
\newcommand{\R}{\mathbb{R}}
\newcommand{\Q}{\mathbb{Q}}
\newcommand{\N}{\mathbb{N}}
\newcommand{\floor}[1]{\left\lfloor #1\right\rfloor}
\newcommand{\C}{\mathbb{C}}
\DeclareMathOperator{\Diam}{diam}
\newcommand{\diam}[1]{\Diam\left(#1\right)}
\renewcommand\qedsymbol{$\blacksquare$}

\newcommand{\Mod}[1]{\ (\bmod\ #1)}


\begin{document}
\title{21-127 Homework 8
}
\author{Christian Broms \\ Section J}
\date{\today}
\maketitle

Complete the following problems. Fully justify each response.

NOTE: due to the Spring Break, this homework set is a bit longer than is typical. You only need to turn in those problems marked with (*).

\begin{enumerate}


\item (*) Let $X$ be a finite set, and suppose there is a surjection $f:X\to Y$. Prove that $|X|\geq |Y|$. 
\begin{proof}

Assume by contradiction, that $|X| < |Y|$. 
If $f$ is surjective, then $\forall y \in Y$, $\exists x \in X$ such that $f(x) = y$. Because every $y \in Y$ must be mapped to an $x \in X$, and $|X| < |Y|$, there will not be enough elements in $X$ to map to every $y$, and we have arrived at a contradiction. Since $f$ is a function and surjective, each $y$ must have an $x \in X$, so it follows that $|X|\geq |Y|$.
\end{proof}

\item (*) Let \[X_2=\{n\ | 1\leq n\leq 200, n=k^2\ \exists k\in \Z\},\] \[X_3=\{n\ | \ 1\leq n\leq 200, n=k^3\ \exists k\in \Z\},\] and \[X_4 = \{n\ | \ 1\leq n\leq 200, n=k^4\ \exists k\in \Z\}.\] Determine $|X_2\cup X_3\cup X_4|$.

The set of $X_2$ is defined as $\{1, 4, 9, 16, 25, 36 \dots 196\}$, so it has the squares of $k = 1$ through $k = 14$. Thus $|X_2| = 14$. The set $X_3$ is defined as $\{1, 8, 27, 64, 125\}$, with $k = 1$ through $k = 5$. So $|X_3| = 5$. Finally, the set $X_4$ is defined as $\{1, 16, 81\}$. So it has values for $k = 1$ through $k = 3$, and $|X_4| = 3$. 

To calculate $|X_2\cup X_3\cup X_4|$, we use the inclusion-exclusion rule to get $|X_2\cup X_3\cup X_4| = |X_2| + |X_3| + |X_4| - |X_2 \cap X_3| - |X_2 \cap X_4| - |X_3 \cap X_4| + |X_2 \cap X_3 \cap X_4|$.  Now, we can simply insert the cardinalities. So, $14 + 5 + 3 - 2 - 3 - 1 + 1 = 17$. Thus, $|X_2\cup X_3\cup X_4| = 14$. 


\item (*) Let $X=\{ (a_1, a_2, \dots, a_n)\ | \ a_i\in\{0,1\}\forall i\} = \{0,1\}^n$. These are sometimes called bitstrings of length $n$.
\begin{enumerate}
\item Show that there is a bijection between $X$ and $\{f:[n]\to\{0,1\}\}$, the set of functions from $[n]$ to $\{0,1\}$

\begin{proof}
The function works by taking a bitstring of length $n$ and mapping each index to 0 or 1. So the bitstring can be written as $(a_1, a_2 \dots a_i)$, where each $a_i$ is mapped to $\{0, 1\}$ by $f(a_i)$. Let $Y = \{f:[n]\to\{0,1\}\}$ and $g:X \to Y$, where $g(x) = f([n])$, and $x=\{0, 1\}^n$. We can show that this function is injective. Let $g(a) = g(b)$ for some $a, b \in \Z$. Then $\{0, 1\}^a =  \{0, 1\}^b$, and $a = b$. Thus $g$ is injective. 

Now, we will show $g$ is surjective. For all $f([n])$ there exists some $\{0, 1\}^n$. Thus, for all $f([n])$
there exists some $x \in X$. 

Since $g$ is injective and surjective, it must be bijective. Thus, there is a bijection between $X$ and $\{f:[n]\to\{0,1\}\}$.
\end{proof}

\item Show that there is a bijection between $X$ and $\mathcal{P}([n])$.

\begin{proof}
Let $Y = \{f:[n]\to\{0,1\}\}$ and define $g:Y \to \mathcal{P}([n])$ and $g(f) = \{n \in [n]\ |\ f(n) = 1\}$.

Thus, to show $g$ is injective, we set $g(f_1) = g(f_2)$ for some $f_1, f_2$. Then $f_1(n) = 1$ when $f_2(n) = 1$ and $f_1(n) = 0$ when $f_2(n) = 0$. Thus, $f_1 = f_2$ and $g$ is injective. 

Show $g$ surjective. Let $A \in \mathcal{P}([n])$. For some $f$, $g(f)$ maps every element in $A$ to 1 or 0. So $g$ is surjective. 

Thus, there exists a bijection between $X$ and $\mathcal{P}([n])$.
\end{proof}


\item Determine $|X|$.

There are 2 options for each of the $n$ elements of $X$, so $|X| = 2^n$
\end{enumerate}

\item (*) Let $X$ and $Y$ be finite sets. Define $X^Y = \{f:Y\to X\}$, the set of functions from $Y$ to $X$. Prove that $|X^Y|=|X|^{|Y|}$.

\begin{proof}
Since $X, Y$ are finite sets, let $|X| = n$, $|Y| = m$ for some $n, m \in \N$. So clearly, $|X|^{|Y|} = n^m$. Then $X = \{x_1, x_2 \dots x_n\}$ and $Y = \{y_1, y_2 \dots y_m\}$. We need to map each $x \in X$ with some $y \in Y$. Each of the elements in $X$ can be mapped to $Y$ in $m$ different ways. Because there are $n$ elements in X, $|X^Y| = n^m$
\end{proof}

\item (*) Let $n, k\in \N$ with $n\geq k$. Prove, by counting in 2 ways, that $k{n\choose k} = (n-k+1){n\choose k-1}$.

\begin{proof}
Suppose we wish to select from a group of $n$ people a committee of $k$ people with a president (so there are $k-1$ members plus one president on the committee). There are two ways to do this. First, we could select the $k$ people for the committee from the $n$ total people, and then select the president from the committee of $k$ people so we have ${k \choose 1}{n\choose k} = k{n\choose k}$.

On the other hand, we could first select the committee with $k - 1$ members from the $n$ total people. We then choose the president from the remaining group of people, which is now $n - k + 1$ in size. Using this selection technique we get ${n-k+1 \choose 1}{n\choose k-1} = (n-k+1){n\choose k-1}$. 

Thus, since both sides of the equality count the same set, they are equal. 
\end{proof}

\item (*) How many subsets of $[20]$ contain a multiple of 4? Prove that your answer is correct.

\begin{proof}
We have previously proven that $ | \mathcal{P}([n]) | = 2^n$. We can use this to attain the fact that $| \mathcal{P}([20]) | = 2^{20}$, so there are $2^{20}$ possible subsets of $\{1, 2, 3, \ldots 20\}$. We write the set of possible multiples of 4 as $\{4, 8, 12, 16, 20\}  = A$. There are $2^5$ possible subsets of A. We can also subtract the number of non-multiples of 4 from the set of $[20]$, so $2^{20} - 2^{15} = 2^5 = 32$.
%Define $X_k = \{A \subseteq [12]\ |\ k \in A\}$. We wish to know $|X_4 \cup X_8 \cup X_{12} \cup X_{16} \cup X_{20}|$. Note that each $X_k$ is th powerset of $[20]$ minus the $k$ element. So for example  $ |X_4 | = \mathcal{P}([20] / \{4\}) $. So we can calculate $|X_4 \cup X_8 \cup X_{12} \cup X_{16} \cup X_{20}|$ as $\sum\limits_{i = 1}^{5}X_{4i} - \sum\limits_{i \neq j, i,j \in A}^{}|X_i \cap X_j| + \sum\limits_{i \neq j \neq k, i,j, k \in A}^{}|X_i \cap X_j \cap X_k| + \sum\limits_{i \neq j \neq k \neq \ell, i,j, k, \ell \in A}^{}|X_i \cap X_j \cap X_k \cap X_\ell|- |X_4 \cap X_8 \cap X_{12} \cap X_{16} \cap X_{20}|$. 
\end{proof}

\item (*) Let $f:X\to Y$ be a bijection. Prove that $X$ is countably infinite if and only if $Y$ is countably infinite.

\begin{proof}
Assume $X$ is countably infinite. So there exists a bijection $g:\N \to X$. We can therefore create a composition $g \circ f:\N \to Y$. Since there exists a bijection from the naturals to $Y$, we conclude $Y$ is countably infinite.

Assume $Y$ is countably infinite. So there exists $f^{-1}: Y \to X$. Since $Y$ is countably infinite, there is a function $g:\N \to Y$. So if we take $f^-1 \circ g: \N \to X$, there is clearly a bijection between the natural numbers and $X$, so it is countably infinite.  
\end{proof}

\item (*) Let $X$ be a finite set. Show that $\N^X=\{f:X\to \N\}$ is countably infinite.

\begin{proof}
We proceed by induction on $X$. 

Base Case: $|X| = 1$. $X$ can be mapped to any value in $\N$ by specifying some function $f$, where $f(x) = a$, $a \in \N$. Since there are a countably infinite number of possible values to set $a$ to, there are countably infinite possible functions $f$ mapping $x$ to $a$. There are therefore $|X|^1$ possiblities, which is countably infinite. 

Inductive Hypothesis: $\N^X=\{f:X\to \N\}$, $\N^X$ is countably infinite. We will show that $\N^{X + 1}$ is countably infinite for $N^{X + 1} = f:\{A \to \N\}$, where $A$ is a set with the cardinality $|X| + 1$. For $\N^{X + 1}$ to be countably infinite, there must be some $g:\N^{X + 1} \to \N$, and $g$ is bijective. $|\N^{X + 1}|$ can be found by $\N^X \times \N$, since there are $\N^X$ different functions that map $X$ to $\N$. Now we must show that $\N^X \times \N$ forms a bijection with $\N$. We know $\N^X$ is countably infinite by IH and $\N$ is countably infinite as well. The cartesian product of two countably infinite sets is known to be countably infinite as well (proved in book), so we know $\N^X \times \N$ is countably infinite, so we conclude that $\N^{X + 1}$ is countably infinite.  
\end{proof}

\end{enumerate}

\end{document}